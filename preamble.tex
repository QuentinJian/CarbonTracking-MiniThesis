%%%%%%%%%%%字型與中文設定%%%%%%%%%%%
\usepackage{fontspec}
\usepackage{xeCJK}
\setCJKmainfont[AutoFakeBold=3,AutoFakeSlant=.2]{[PMingLiU.ttf]}
\setCJKmonofont{[PMingLiU.ttf]}
\setCJKsansfont{[PMingLiU.ttf]}
\setmainfont[AutoFakeBold=3,AutoFakeSlant=.2]{[times.ttf]}
\setsansfont[AutoFakeBold=3,AutoFakeSlant=.2]{[times.ttf]}
\setmonofont{[times.ttf]}
\newfontfamily\ming[AutoFakeBold=3,AutoFakeSlant=.2]{[PMingLiU.ttf]}
\newfontfamily\timesnew[AutoFakeBold=3,AutoFakeSlant=.2]{[times.ttf]}
\XeTeXlinebreaklocale "zh" %這兩行一定要加,中文才能自動換行
\XeTeXlinebreakskip = 0pt plus 1pt %這兩行一定要加,中文才能自動換行
\usepackage{newtxtext,newtxmath}

%%%%%%%%%%%邊界設定、文字格式%%%%%%%%%%%
\usepackage[margin=2cm, a4paper]{geometry}
\usepackage{indentfirst} 
\setlength{\parindent}{2em}
\usepackage{setspace}
\renewcommand{\baselinestretch}{1.0}
\setlength{\parskip}{1em}
\usepackage[utf8]{inputenc}
\usepackage{changepage} %adjustwidth

%%%%%%%%%%%頁首頁尾%%%%%%%%%%%
\usepackage{fancyhdr}
\pagestyle{fancy}
\renewcommand\headrulewidth{0pt}
\lhead{}
%小論文篇名的部分放在 main 供修改
\rhead{}
\lfoot{}
\cfoot{\fontsize{10pt}{\baselineskip}\selectfont \thepage}
\rfoot{}

% 定義 \XeTeX 符號:
\def\reflect#1{{\setbox0=\hbox{#1}\rlap{\kern0.5\wd0
  \special{x:gsave}\special{x:scale -1 1}}\box0 \special{x:grestore}}}
\def\XeTeX{\leavevmode
  \setbox0=\hbox{X\lower.5ex\hbox{\kern-.15em\reflect{E}}\kern-.1667em \TeX}%
  \dp0=0pt\ht0=0pt\box0 }
\def\XeLaTeX{\leavevmode
  \setbox0=\hbox{X\lower.5ex\hbox{\kern-.15em\reflect{E}}\kern-.0833em \LaTeX}%
  \dp0=0pt\ht0=0pt\box0 }

%%%%%%%%%%%自定義中文數字%%%%%%%%%%%
\newcommand{\CCNumberA}[1]{\ifcase#1\or{壹}\or{貳}\or{參}\or{肆}\or{伍}\or{陸}\or{柒}\or{捌}\or{玖}\or{拾}
    \or{拾壹}\or{拾貳}\or{拾參}\or{拾肆}\or{拾伍}\or{拾陸}\or{拾柒}\or{拾捌}\or{拾玖}\or{貳拾}
    \or{貳拾壹}\or{貳拾貳}\or{貳拾參}\or{貳拾肆}\or{貳拾伍}\or{貳拾陸}\or{貳拾柒}\or{貳拾捌}\or{貳拾玖}\or{參拾}
    \fi}
\newcommand{\CCNumberB}[1]{\ifcase#1\or{一}\or{二}\or{三}\or{四}\or{五}\or{六}\or{七}\or{八}\or{九}\or{十}
     \or{十一}\or{十二}\or{十三}\or{十四}\or{十五}\or{十六}\or{十七}\or{十八}\or{十九}\or{二十}
     \or{二十一}\or{二十二}\or{二十三}\or{二十四}\or{二十五}\or{二十六}\or{二十七}\or{二十八}\or{二十九}\or{三十}
     \or{三十一}\or{三十二}\or{三十三}\or{三十四}\or{三十五}\or{三十六}\or{三十七}\or{三十八}\or{三十九}\or{四十}
     \or{四十一}\or{四十二}\or{四十三}\or{四十四}\or{四十五}\or{四十六}\or{四十七}\or{四十八}\or{四十九}\or{五十}
     \or{五十一}\or{五十二}\or{五十三}\or{五十四}\or{五十五}\or{五十六}\or{五十七}\or{五十八}\or{五十九}\or{六十}
     \or{六十一}\or{六十二}\or{六十三}\or{六十四}\or{六十五}\or{六十六}\or{六十七}\or{六十八}\or{六十九}\or{七十}   
    \fi}

%%%%%%%%%%%章節階層設定%%%%%%%%%%%
\usepackage{titlesec,titletoc}
\setcounter{secnumdepth}{4} %設定計數器使 paragraph 也有編號

\titleformat{\section}                      % #1.改變 section 的指令 
[hang]                                      % #2.將標號及標題視為同一個段落
{\normalfont}                               % #3.標號標題的字形格式
{\\ \CCNumberA{\arabic{section}}、}         % #4.標號
{0pc}                                       % #5.標號標題間隔
{}                                          % #6.標題前之指令
[]                                          % #7.標題後之指令、選擇性輸入

\titleformat{\subsection}                   % #1.改變 subsection 的指令 
[hang]                                      % #2.將標號及標題視為同一個段落
{\normalfont}                               % #3.標號標題的字形格式
{\\ \CCNumberB{\arabic{subsection}}、}      % #4.標號
{0pc}                                       % #5.標號標題間隔
{}                                          % #6.標題前之指令
[]                                          % #7.標題後之指令、選擇性輸入
   
\titleformat{\subsubsection}                % #1.改變 subsubsection 的指令 
[hang]                                      % #2.將標號及標題視為同一個段落
{\normalfont}                               % #3.標號標題的字形格式
{(\CCNumberB{\arabic{subsubsection}})}  % #4.標號
{0pc}                                       % #5.標號標題間隔
{}                                          % #6.標題前之指令
[]                                          % #7.標題後之指令、選擇性輸入

\titleformat{\paragraph}                    % #1.改變 paragraph 的指令 
[hang]                                      % #2.將標號及標題視為同一個段落
{\normalfont}                               % #3.標號標題的字形格式
{\arabic{paragraph}、}                      % #4.標號
{0pc}                                       % #5.標號標題間隔
{}                                          % #6.標題前之指令
[]                                          % #7.標題後之指令、選擇性輸入

\titlespacing{\section} {0em}{0em}{0em}
\titlespacing{\subsection} {2em}{0em}{0em}
\titlespacing{\subsubsection} {4em}{0em}{0em}
\titlespacing{\paragraph} {6em}{0em}{0em}

%%%%%%%%%%%原文照刊環境中文處理(\begin{verbatim})%%%%%%%%%%%
\makeatletter
\def\verbatim@font{\rmfamily} %如果使用roman字体族,将sffamily改成rmfamily
\makeatother
%%%%%%%%%%%列表%%%%%%%%%%%
\usepackage{enumerate}

%%%%%%%%%%%超連結%%%%%%%%%%%
\usepackage{xurl}
\usepackage[bookmarksopen,pdfstartview=FitH,breaklinks=true,
linkcolor=black,anchorcolor=black,citecolor=black,hidelinks]{hyperref}
\urlstyle{same}
%%%%%%%%%%%圖片%%%%%%%%%%%
\usepackage{graphicx}
\usepackage{caption}
\usepackage{float}
\usepackage{subfigure}

%%%%%%%%%%%翻譯中文%%%%%%%%%%%
\renewcommand{\tablename}{表}
\renewcommand{\figurename}{圖}
\renewcommand{\thefigure}{\CCNumberB{\arabic{figure}}}
%%%%%%%%%%%本模版獨特的 new command%%%%%%%%%%%
%打 LaTeX 打不出來的特殊符號
\newcommand\apo{\textquotesingle}

\newcommand{\coverpage}[6]
{
	\begin{titlepage}
		\begin{center}
			投稿類別:#1\\[7cm]
			篇名:\\
			#2\\[6cm]
			作者:\\
		    \if\relax\detokenize{#3}\relax
            
            \else
            #3 \\
            \fi
		    \if\relax\detokenize{#4}\relax
            
            \else
            #4 \\
            \fi
		    \if\relax\detokenize{#5}\relax
            
            \else
            #5 \\
            \fi
			\vfill
			指導老師:\\
			#6
			\end{center}
	\end{titlepage}
}

%頁首
\newcommand{\centerhead}[1]
{
    \chead{\fontsize{10pt}{\baselineskip}\selectfont #1}
}

\newcommand{\h}[1]{\section{#1}}        %令section=h
\newcommand{\hh}[1]{\subsection{#1}}    %令subsection=hh
\newcommand{\hhh}[1]{\subsubsection{#1}}%令subsubsection=hhh
\newcommand{\hhhh}[1]{\paragraph{#1}}   %令paragraph=hhhh

%段落內容及文字縮排(content & text)
\newcommand{\hhtext}[1]
{
    \if\relax\detokenize{#1}\relax
    
    \else
        \begin{adjustwidth}{2em}{0em}
        \hspace*{21pt}#1
        \end{adjustwidth}
    \fi
}
\newcommand{\subsectiontext}[1]
{
    \if\relax\detokenize{#1}\relax
    
    \else
        \begin{adjustwidth}{2em}{0em}
        \hspace*{21pt}#1
        \end{adjustwidth}
    \fi
}
\newcommand{\hhhtext}[1]
{
    \if\relax\detokenize{#1}\relax
    
    \else
        \begin{adjustwidth}{4em}{0em}
        \hspace*{21pt}#1
        \end{adjustwidth}
    \fi
}
\newcommand{\subsubsectiontext}[1]
{
    \if\relax\detokenize{#1}\relax
    
    \else
        \begin{adjustwidth}{4em}{0em}
        \hspace*{21pt}#1
        \end{adjustwidth}
    \fi
}
\newcommand{\hhhhtext}[1]
{
    \if\relax\detokenize{#1}\relax
    
    \else
        \begin{adjustwidth}{6em}{0em}
        \hspace*{21pt}#1
        \end{adjustwidth}
    \fi
}

%%%%%%%%%%%%%%%Reference%%%%%%%%%%%%%%%%%%%%%
%無特定格式
\newcommand{\nonetype}[1]
{
    #1
}
%書籍類
\newcommand{\zhbook}[4]
{
    #1(#2)。\textbf{#3}。#4。
}
\newcommand{\enbook}[4]
{
    \if\relax\detokenize{#1}\relax
        \textit{#3}. (#2). #4.
    \else
        #1. (#2). \textit{#3}. #4.
    \fi
}
\newcommand{\ezhbook}[4]
{
    #1(#2)。\textbf{#3}。\url{#4}
}
\newcommand{\eenbook}[4]
{
    #1. (#2). \textit{#3}. \url{#4}
}
%期刊論文類
\newcommand{\zhjournal}[7]
{
    \if\relax\detokenize{#5}\relax
        #1(#2)。#3。\textbf{#4,#6},#7。
    \else
         #1(#2)。#3。\textbf{#4,#5}(#6),#7。
    \fi
}
\newcommand{\enjournal}[7]
{
    \if\relax\detokenize{#5}\relax
        #1. (#2). #3. \textit{#4, #6}, #7.
    \else
        #1. (#2). #3. \textit{#4,#5}(#6), #7.
    \fi
}
\newcommand{\ezhjournal}[8]
{
    \if\relax\detokenize{#5}\relax
        #1(#2)。#3。\textbf{#4,#6},#7。\url{#8}
    \else
        #1(#2)。#3。\textbf{#4,#5}(#6),#7。\url{#8}
    \fi
}
\newcommand{\eenjournal}[8]
{
    \if\relax\detokenize{#5}\relax
        #1. (#2). #3. \textit{#4, #6}, #7. \url{#8}
    \else
        #1. (#2). #3. \textit{#4,#5}(#6), #7. \url{#8}
    \fi
}
%文集論文類
\newcommand{\zhanthology}[7]
{
    #1(#2)。#3。#4:\textbf{#5}(#6)。#7。
}
\newcommand{\enanthology}[7]
{
    #1.(#2). #3. #4. \textit{#5}, (#6). #7.
}
%博(碩)士論文
\newcommand{\zhthesis}[5]
{
    #1(#2)。\textbf{#3}。#4:#5。
}
\newcommand{\enthesis}[5]
{
    #1. (#2).\textit{#3}. #4: #5.
}
\newcommand{\ezhthesis}[6]
{
    #1(#2)。\textbf{#3}。#4:#5。\url{#6}
}
\newcommand{\eenthesis}[6]
{
    #1. (#2).\textit{#3}. #4: #5. \url{#6}
}
%報紙文章
\newcommand{\zhnewspaper}[5]
{
    #1(#2)。#3。\textbf{#4},#5。
}
\newcommand{\ennewspaper}[5]
{
    #1. (#2). #3. \textit{#4}, #5.
}
\newcommand{\ezhnewspaper}[5]
{
    #1(#2)。#3。\textbf{#4}。\url{#5}
}
\newcommand{\eennewspaper}[5]
{
    #1. (#2). #3. \textit{#4}. \url{#5}
}
%法規
\newcommand{\law}[2]
{
    #1(#2)。
}
%網路相關資源
\newcommand{\simpleinternet}[4]
{
    #1(#2)。#3。\url{#4}
}
\newcommand{\internet}[4]
{
    #1(#2)。#3。\url{#4}
}
\newcommand{\nodateinternet}[4]
{
    #1(無日期)。#2。#3,取自 \url{#4}
}
\newcommand{\youtube}[4]
{
    #1(#2)。#3[影片]。YouTube。\url{#4}
}
\newcommand{\facebook}[5]
{
    #1(#2)。#3[#4]。Facebook。\url{#5}
}
\newcommand{\blog}[4]
{
    #1(#2)。#3[部落格文章]。\url{#4}
}
